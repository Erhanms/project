\chapter{Metode}
\paragraph{}I kapittel 3 dreier det seg om metode. Med metode menes hvilke arbeidsmåter som er tatt i bruk for innhentingen av informasjon. I dette tilfellet inkluderer dette nettsøk, intervju og en spørreundersøkelse. Metode beskriver hva gruppen gjorde for å finne den informasjonen som blir benyttet for å skape en troverdig rapport.
\section{Innhenting av informasjon}
\paragraph{} \gruppen arbeidet med oppgaven på en rekke ulike måter for å finne informasjon og data for løsningene og analysene. Innsamlingen av informasjon foregikk på flere måter, for det første så arrangerte gruppen 2 møter med daglig leder for Katoplast for å innhente så mye informasjon som mulig før gruppen gikk i gang med prosjektet. Under disse møtene med Katoplast, dokumenterte gruppen det viktigste av informasjon, samt også hva det er bedriften ser etter i rapporten og de løsningene rapporten skal fremstille som konklusjon.

\paragraph{} I tillegg til å gjennomføre møter med Katoplast, var det også betydelig å gjennomføre flere, grundige søk på internett og i bøker for å finne de mest optimale løsningene og det viktigste av informasjon som kreves for å skrive gode analyser. Innhentingen av informasjon ble delt i 4, hvor hvert gruppemedlem fikk et tema de skulle fokusere på når de skulle hente ut informasjon. Informasjonsinnhentingen krevde samtidig at kildene som ble gjennomsøkt var troverdige og at disse ble analysert og sammenlignet med andre kilder for å finne korrekt og troverdig informasjon. 

\paragraph{} En rekke informasjon lå i tidligere dokumenter fra faget "Webutvikling" hvor gruppen allerede hadde notert troverdig og verdifull informasjon om servere og Sky løsninger. I disse dokumentene var det også rikelig med informasjon rundt terminologi og hvordan internett vil utvikle seg fremover. Dette er informasjon som kan være vesentlig for prosjektet og analysene å ha med, og dermed ble denne informasjonen også inkludert i prosjektet.

\paragraph{} I tillegg til dette, valgte gruppen å sende et dokument med en rekke spørsmål til daglig leder av Katoplast, Trond Kjellin, slik at han kunne svare på noen få spørsmål som gruppen lurte på. Dette var en rekke spørsmål som ville hjelpe gruppen med å rangere  hvilke funksjoner og deler ved løsningene som skulle velges, og hvilke deler ved analysene det skulle legges mer vekt på enn andre. Samtidig var disse spørsmålene også vesentlige for at gruppen kunne utvikle en situasjonsanalyse av den nåværende situasjonen og problemene som bedriften opplever. 

\section{Intervju}
\paragraph{} Gruppen oppsøkte Katoplast via e-post og ønsket å sette opp et intervju. 10.02.2017 hadde prosjektgruppen det første møte med bedriften. Gruppen hadde en velkomstrunde i bedriften og fikk kartlagt hvordan bedriften jobber daglig. Det første møte med bedriften gikk ut på å finne ut av hva gruppen kunne bistå med for Katoplast. Etter endt møte sto prosjektgruppen igjen med problemstillinger som ERP, administrative oppgaver og servere. Det første intervjuet ga gruppen et bredt spekter av problemer som gruppen kunne ta for seg. Gruppen valgte å fokusere på å hjelpe bedriften med deres server problemer for å kutte ned på prosjektet. Dette var viktig med tanke på tidsrommet og hvilke ressurser gruppen hadde tilgang til.

\paragraph{} På det neste møte med daglig leder Kjellin, var det vesentlig for gruppen å kartlegge Katoplast sin server situasjon i dag, hva slags problemer de opplever, både interne og eksterne, og ikke minst finne ut av hva Katoplast ønsker å oppnå med prosjektet. Dette intervjuet var avgjørende for at gruppen kunne begrense arbeidet til mindre løsninger, fremfor å skrive en bred analyse om mange løsninger.

\section{Nettsøk}
\paragraph{} Som nevnt tidligere, så delte gruppen  opp nettsøket i 4 deler; Interne servere, eksterne servere, Cloud og Hybrid-løsning. Hvert gruppemedlem fikk tildelt hver sin del av disse 4 delene, og internett søket ble satt i gang når medlemmene var klare.

\paragraph{} For interne servere gjennomførte prosjektgruppen en rekke søk på nettet hvor gruppen la vekt på norske leverandører av servere. Det var nødvendig å finne leverandører som hadde akseptable priser, og ikke minst god service slik at det ville være enklere å få ut informasjon.

\paragraph{} For Hybride-løsninger ble det først og fremst søkt gjennom hvilke leverandører som tilbyr en Hybrid Cloud, og hvilke av disse leverandørene som virker mest troverdige. Gruppen benyttet ulike nettsider for å finne den beste informasjonen. Gruppen gikk gjennom en rekke med anmeldelser og sammenligninger av ulike hybride løsninger og hvordan disse fungerte sammen. Den mest fundamentale ressursen for informasjon var leverandørenes egne nettsider og informasjon. Det var her gruppen fant beste kvalitet på informasjonen og ikke minst var det troverdig informasjon. 


