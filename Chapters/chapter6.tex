\chapter{Konklusjon}
\paragraph{} Konklusjonen er det siste kapittelet i rapporten, og kapittelet oppsummerer det prosjektgruppen har funnet ut av gjennom analysene og forskningen som er blitt gjennomført. Konklusjonen vil også presentere en anbefaling for en løsning til bedriften som de kan få nytte av. 

\section{Anbefaling}
\paragraph{} Prosjektet har tatt for seg ulike serverløsninger for både interne, eksterne og hybride servere -og har analysert muligheter innenfor hver av disse områdene. Med et fokus på kravspesifikasjonene og avgrensningene som bedriften la ned, har prosjektgruppen klart å analysere og rapportere om ulike løsninger som kan løse Katoplast sitt server problem. Det er imidlertid slik at ikke alle disse løsningene er perfekte, og hver og en av disse har sine fordeler og ulemper som kan være interessante for Katoplast å forstå.

\paragraph{} Med tanke på økonomien vil Katoplast få betraktelig mindre kostnader når det gjelder kjøp og drifting av serverne. PC Hjelp AS tilbyr en server for 10 072 kr og konsulent arbeid for 3 160 kr. I motsetning til Borg Svakstrøm som tilbød serverne sine for 84 900 kr samt installasjon og oppsett av serverne på 7 dager for 41 000 kr. Dette er en merkbar forskjell på prisene. Dersom Katoplast ikke ønsker å fjerne sine gamle servere, foreslår prosjektgruppen at Katoplast heller beholder de gamle serverne sine -og benytter disse som backup servere, mens de nye serverne fra PC Hjelp AS benyttes som standard servere.

\paragraph{} Etter en grundig evaluering av alle løsningene, en evaluering av katoplast sin nåværende situasjon, deres ønsker og de arbeidskravene som er stilt for prosjektet så har gruppens valg falt på PC Hjelp AS. PC Hjelp AS tilbyr gruppen en svært rimelig løsning som er billig, effektiv og som møter ønskene til katoplast. Dette er en intern server som PC Hjelpen tilbyr med cirka 2TB lagringsplass. Operativ Systemet er Windows, som er nettopp det Katoplast ønsket. I tillegg til dette tilbyr PC Hjelp AS konsulent-arbeid for oppsett av serveren. Dette vil ta mellom 1-2 dager og vil koste Katoplast 3 160 kr. Valget til gruppen falt på denne løsningen ettersom den var svært rimelig for økonomien, samtidig som at det er en intern server noe Katoplast hadde som et sterkt ønske. 